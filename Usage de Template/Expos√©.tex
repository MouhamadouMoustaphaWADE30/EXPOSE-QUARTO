% Options for packages loaded elsewhere
\PassOptionsToPackage{unicode}{hyperref}
\PassOptionsToPackage{hyphens}{url}
\PassOptionsToPackage{dvipsnames,svgnames,x11names}{xcolor}
%
\documentclass[
  letterpaper,
  DIV=11,
  numbers=noendperiod]{scrartcl}

\usepackage{amsmath,amssymb}
\usepackage{iftex}
\ifPDFTeX
  \usepackage[T1]{fontenc}
  \usepackage[utf8]{inputenc}
  \usepackage{textcomp} % provide euro and other symbols
\else % if luatex or xetex
  \usepackage{unicode-math}
  \defaultfontfeatures{Scale=MatchLowercase}
  \defaultfontfeatures[\rmfamily]{Ligatures=TeX,Scale=1}
\fi
\usepackage{lmodern}
\ifPDFTeX\else  
    % xetex/luatex font selection
\fi
% Use upquote if available, for straight quotes in verbatim environments
\IfFileExists{upquote.sty}{\usepackage{upquote}}{}
\IfFileExists{microtype.sty}{% use microtype if available
  \usepackage[]{microtype}
  \UseMicrotypeSet[protrusion]{basicmath} % disable protrusion for tt fonts
}{}
\makeatletter
\@ifundefined{KOMAClassName}{% if non-KOMA class
  \IfFileExists{parskip.sty}{%
    \usepackage{parskip}
  }{% else
    \setlength{\parindent}{0pt}
    \setlength{\parskip}{6pt plus 2pt minus 1pt}}
}{% if KOMA class
  \KOMAoptions{parskip=half}}
\makeatother
\usepackage{xcolor}
\usepackage{soul}
\setlength{\emergencystretch}{3em} % prevent overfull lines
\setcounter{secnumdepth}{-\maxdimen} % remove section numbering
% Make \paragraph and \subparagraph free-standing
\ifx\paragraph\undefined\else
  \let\oldparagraph\paragraph
  \renewcommand{\paragraph}[1]{\oldparagraph{#1}\mbox{}}
\fi
\ifx\subparagraph\undefined\else
  \let\oldsubparagraph\subparagraph
  \renewcommand{\subparagraph}[1]{\oldsubparagraph{#1}\mbox{}}
\fi


\providecommand{\tightlist}{%
  \setlength{\itemsep}{0pt}\setlength{\parskip}{0pt}}\usepackage{longtable,booktabs,array}
\usepackage{calc} % for calculating minipage widths
% Correct order of tables after \paragraph or \subparagraph
\usepackage{etoolbox}
\makeatletter
\patchcmd\longtable{\par}{\if@noskipsec\mbox{}\fi\par}{}{}
\makeatother
% Allow footnotes in longtable head/foot
\IfFileExists{footnotehyper.sty}{\usepackage{footnotehyper}}{\usepackage{footnote}}
\makesavenoteenv{longtable}
\usepackage{graphicx}
\makeatletter
\def\maxwidth{\ifdim\Gin@nat@width>\linewidth\linewidth\else\Gin@nat@width\fi}
\def\maxheight{\ifdim\Gin@nat@height>\textheight\textheight\else\Gin@nat@height\fi}
\makeatother
% Scale images if necessary, so that they will not overflow the page
% margins by default, and it is still possible to overwrite the defaults
% using explicit options in \includegraphics[width, height, ...]{}
\setkeys{Gin}{width=\maxwidth,height=\maxheight,keepaspectratio}
% Set default figure placement to htbp
\makeatletter
\def\fps@figure{htbp}
\makeatother

\KOMAoption{captions}{tableheading}
\makeatletter
\makeatother
\makeatletter
\makeatother
\makeatletter
\@ifpackageloaded{caption}{}{\usepackage{caption}}
\AtBeginDocument{%
\ifdefined\contentsname
  \renewcommand*\contentsname{Table of contents}
\else
  \newcommand\contentsname{Table of contents}
\fi
\ifdefined\listfigurename
  \renewcommand*\listfigurename{List of Figures}
\else
  \newcommand\listfigurename{List of Figures}
\fi
\ifdefined\listtablename
  \renewcommand*\listtablename{List of Tables}
\else
  \newcommand\listtablename{List of Tables}
\fi
\ifdefined\figurename
  \renewcommand*\figurename{Figure}
\else
  \newcommand\figurename{Figure}
\fi
\ifdefined\tablename
  \renewcommand*\tablename{Table}
\else
  \newcommand\tablename{Table}
\fi
}
\@ifpackageloaded{float}{}{\usepackage{float}}
\floatstyle{ruled}
\@ifundefined{c@chapter}{\newfloat{codelisting}{h}{lop}}{\newfloat{codelisting}{h}{lop}[chapter]}
\floatname{codelisting}{Listing}
\newcommand*\listoflistings{\listof{codelisting}{List of Listings}}
\makeatother
\makeatletter
\@ifpackageloaded{caption}{}{\usepackage{caption}}
\@ifpackageloaded{subcaption}{}{\usepackage{subcaption}}
\makeatother
\makeatletter
\@ifpackageloaded{tcolorbox}{}{\usepackage[skins,breakable]{tcolorbox}}
\makeatother
\makeatletter
\@ifundefined{shadecolor}{\definecolor{shadecolor}{rgb}{.97, .97, .97}}
\makeatother
\makeatletter
\makeatother
\makeatletter
\makeatother
\ifLuaTeX
  \usepackage{selnolig}  % disable illegal ligatures
\fi
\IfFileExists{bookmark.sty}{\usepackage{bookmark}}{\usepackage{hyperref}}
\IfFileExists{xurl.sty}{\usepackage{xurl}}{} % add URL line breaks if available
\urlstyle{same} % disable monospaced font for URLs
\hypersetup{
  pdftitle={EXPOSE QUARTO},
  colorlinks=true,
  linkcolor={blue},
  filecolor={Maroon},
  citecolor={Blue},
  urlcolor={Blue},
  pdfcreator={LaTeX via pandoc}}

\title{EXPOSE QUARTO}
\author{}
\date{}

\begin{document}
\maketitle
\ifdefined\Shaded\renewenvironment{Shaded}{\begin{tcolorbox}[interior hidden, borderline west={3pt}{0pt}{shadecolor}, enhanced, frame hidden, sharp corners, breakable, boxrule=0pt]}{\end{tcolorbox}}\fi

\renewcommand*\contentsname{Table of contents}
{
\hypersetup{linkcolor=}
\setcounter{tocdepth}{6}
\tableofcontents
}
\newpage{}

\hypertarget{resume}{%
\section{RESUME}\label{resume}}

La migration est un fléau qui touche particulièrement les jeunes ces
dernières années et le plus souvent à la recherche de
l\textquotesingle emploi. Ce problème ne peut être résolu
qu\textquotesingle en s\textquotesingle attaquant à ces causes et en
identifiant les caractéristiques des migrants à la recherche de
l\textquotesingle emploi. Ainsi, l\textquotesingle objectif de cette
présente étude consistait à appréhender les déterminants de cette
migration pour les jeunes sénégalais en passant par une observation de
leur répartition à travers certaines variables. Autrement dit, il
s\textquotesingle agissait d\textquotesingle identifier les jeunes
sénégalais qui ont un risque plus élevé de migrer pour de
l\textquotesingle emploi. Pour ce faire, les données de cette étude ont
été tirées de la phase pilote du 5e Recensement Général de
l\textquotesingle Habitat et de la Population au Sénégal (RGPH-5). A cet
effet, la démarche adoptée a consisté, en sus de la revue de littérature
et des présentations du cadre conceptuel et des données, dans un premier
temps à faire une analyse descriptive (univariée, bivariée).
L\textquotesingle objectif recherché dans cette partie est de comprendre
les caractéristiques de notre population d\textquotesingle étude et
leurs profils suivant des variables pouvant influencer la décision de
migrer. Dans un deuxième temps, le profilage des demandeurs
d\textquotesingle emploi a été effectué à l\textquotesingle aide du
modèle logistique en utilisant la variable dépendante binaire « Motif de
migration » et les variables indépendantes suivantes : la région, le
sexe, la situation matrimoniale, le niveau
d\textquotesingle instruction, le milieu de résidence et la profession
au départ du migrant. Les résultats de cette étude montrent
qu\textquotesingle au Sénégal, la proportion jeunes migrants en quête
d\textquotesingle emploi ces cinq dernières années est de 65,67\%. Cette
proportion varie selon le sexe, la situation matrimoniale, la région, ou
d\textquotesingle autres facteurs. Par ailleurs, l\textquotesingle étude
des facteurs prédictifs de la migration pour de
l\textquotesingle emploi, réalisée avec le modèle logistique binaire,
donne un impact statistiquement significatif des variables suivantes :
la région, le sexe, la situation matrimoniale, la tranche
d\textquotesingle âge du migrant, sa profession au départ ainsi que son
niveau d\textquotesingle éducation. Ainsi, on retient que le fait de
n\textquotesingle avoir aucun niveau en termes
d\textquotesingle éducation, le fait d\textquotesingle être de sexe
masculin, le fait d\textquotesingle être dans la tranche
d\textquotesingle âge 25-34 ou encore de résider dans la région de
Ziguinchor et dans le milieu urbain influencent la probabilité de migrer
à la recherche de l\textquotesingle emploi.

\hypertarget{abstract}{%
\section{ABSTRACT}\label{abstract}}

Migration is a scourge that particularly affects young people in recent
years and most often looking for work. This problem can only be solved
by addressing these causes and identifying the characteristics of
migrants seeking employment. Thus, the objective of this study was to
understand the determinants of this migration for young Senegalese
through an observation of their distribution through certain variables.
In other words, it was a question of identifying young Senegalese who
have a higher risk of migrating for employment. To do this, the data for
this study were taken from the pilot phase of the 5th General Census of
Housing and Population in Senegal (CHPS-5). To this end, the approach
adopted consisted, in addition to the literature review and the
presentations of the conceptual framework and the data, first of all in
carrying out a descriptive analysis (univariate, bivariate). The
objective sought in this part is to understand the characteristics of
our study population and their profiles according to variables that can
influence the decision to migrate. In a second step, the profiling of
job seekers was carried out using the logistic model using the binary
dependent variable ``Reason for migration'' and the following
independent variables: region, gender, marital status, level of
education, place of residence and occupation when the migrant leaves.
The results of this study show that in Senegal, the proportion of young
migrants seeking employment over the past five years is 65.67\%. This
proportion varies by gender, marital status, region, or other factors.
In addition, the study of predictive factors of migration for
employment, carried out with the binary logistic model, gives a
statistically significant impact of the following variables: region,
sex, marital status, age group of migrants, his profession at the start
as well as his level of education. Thus, we retain that the fact of
having no level in terms of education, the fact of being male, the fact
of being in the 25-34 age group or even of residing in the region of
Ziguinchor and in the urban environment influence the probability of
migrating in search of employment.

\newpage{}

\hypertarget{introduction-generale}{%
\section{INTRODUCTION GENERALE}\label{introduction-generale}}

\ul{\textbf{CONTEXTE ET JUSTIFICATION}}

La migration est un phénomène universel que l'on retrouve partout et en
tout temps avec une intensité variable. Rares sont les populations et
les territoires qui n'ont pas été le théâtre de flux migratoires. La
migration constitue actuellement une des problématiques majeures de
l'économie mondiale. En effet, aujourd'hui, il n'y a jamais eu autant de
personnes vivant dans un autre pays que celui dans lequel elles sont
nées, selon le rapport de l'Organisation internationale pour les
migrations (OIM), « \emph{État de la migration dans le monde 2022} ». De
plus selon cette même source en 2020, le nombre de migrants dans le
monde était d'environ 281 millions de personnes, soit 51 millions de
plus qu'en 2010, 128 millions de plus qu'en 1990 et plus de trois fois
plus qu'en 1970. Elle demeure au cœur des débats de politique économique
et sociale tant dans les pays de départ que dans les pays d'accueil. En
effet, la nature et l'importance des flux migratoires ont un impact
différent mais significatif sur les économies des pays concernées. Le
phénomène migratoire est donc très complexe et revêt divers aspects
économiques, politiques, culturels et sociaux. Il a certes des
conséquences économiques mais aussi des implications sociale et
culturelles durables tant sur les pays d'accueil que sur les pays
d'origine. Elle prend plus ces derniers temps une forme irrégulière et
est devenue une migration de désespoir au regard des moyens utilisés.
Même si certains d'entre eux arrivent à destination quoiqu'ayant subi
des souffrances, nombreux sont ceux qui meurent en cours de route.

De nos jours, elle touche plus les jeunes qui, face aux situations de
leur pays décident d\textquotesingle émigrer pour diverses raisons. En
effet, selon le Département des Affaires Economiques et Sociales (DAES)
de l\textquotesingle ONU, le nombre estimé de jeunes migrants est passé
de 22,1 millions en 1990 à 31,7 millions en 2020. En 2020, 11,3\% de la
population migrante étaient des jeunes et 2,6\% des jeunes dans le monde
étaient des migrants (DAES, 2020). L\textquotesingle Afrique
n\textquotesingle est pas resté en marge de cette tendance migratoire et
notamment le Sénégal. Par ailleurs, selon le rapport de
l\textquotesingle Afro Barometer paru le 13 Novembre 2020 portant sur
l\textquotesingle émigration des jeunes sénégalais, plus de 50 \% des
jeunes affirment avoir pensé à émigrer dont 30\% qui y ont réfléchi
beaucoup. Parmi les jeunes qui ont au moins « un peu » pensé à émigrer,
la grande majorité s\textquotesingle apprête déjà (9\%) ou planifie de
quitter le pays dans un ou deux ans mais n\textquotesingle ont pas
commencé à se préparer dans ce sens (56\%). Cela montre comment
l\textquotesingle émigration est devenue une option pour les jeunes
sénégalais. De plus, selon cette même source, il a été également
constaté que plus de la moitié des jeunes émigrant migre à la recherche
de travail. Ces jeunes, dans le but de se rendre à ce
qu\textquotesingle ils appellent l\textquotesingle eldorado usent de
tous les moyens. Bien conscients des dangers qu\textquotesingle ils
courent en le faisant, ces derniers se disent qu\textquotesingle il
valle mieux mourir en ayant tenté que de mourir dans des conditions
bizarres au pays. On parle de \emph{Mbëk, barça mba barsakh}
\footnote{Cela pourrait se traduire par \textquotesingle Barcelone ou le
  Paradis, comme si c\textquotesingle était un jihad. Le terme barzakh
  proviendrait de l\textquotesingle arabe qui signifierait la félicité}ou
bien encore de \emph{kaaliss kewdo walla agneere woddunde} \footnote{En
  peul cela signifie littéralement signifie « beaucoup
  d\textquotesingle argent ou tombeau lointain de la patrie »,
  c\textquotesingle est-à-dire « Mieux vaut mourir loin de la misère de
  la communauté que d\textquotesingle assister impuissant devant la
  descente aux Enfers »}. Tous ces termes rappellent dans la conscience
collective des africains de l\textquotesingle ouest
l\textquotesingle épopée guerrière des diamantaires haalpularen et
Soninke notamment, originaires de la vallée du fleuve Sénégal (Mali,
Mauritanie, Sénégal) dans les années 1970. Ainsi, pour les jeunes
sénégalais, émigrer et surtout clandestinement en empruntant des
pirogues est plutôt un choix valorisant (Dr Cheik Oumar Ba et Dr Alfred
Iniss Ndiaye, \emph{L\textquotesingle émigration clandestine
sénégalaise}).

Parallèlement, la situation des jeunes au regard de
l\textquotesingle emploi est plus que préoccupante, plusieurs
indicateurs le confirment : leur insertion sur le marché du travail est
plus que tardive, la précarité de l\textquotesingle emploi et des
revenus est bien réelle, la montée de la pauvreté des jeunes est
choquante sans oublier la pandémie qui est venue aggraver la situation.
En effet, selon l\textquotesingle article du Dr.~Alboury NDIAYE portant
sur l\textquotesingle emploi des jeunes au Sénégal, 60 \% de la
population a moins de 20 ans et les jeunes en âge de travailler
représentent plus de la moitié de la population active. Dans les
prochaines années, l\textquotesingle Afrique en général, sera selon
cette même source, la zone géographique ayant la main
d\textquotesingle œuvre la plus importante en quantité devant la Chine
et l\textquotesingle Inde. L\textquotesingle un des grands défis à
relever aujourd\textquotesingle hui au Sénégal c\textquotesingle est de
permettre à chaque jeune d\textquotesingle exprimer son talent, dans un
pays où près de 2 jeunes sur 3 est au chômage. L\textquotesingle Agence
Nationale de la Statistique et de la Démographie sénégalaise (A.N.S.D)
avait en 2015, estimé que chaque année, plus de 100 000 nouveaux
demandeurs d\textquotesingle emplois entre 15 et 34 ans arrivent sur le
marché du travail. Les chiffres récents, issus de
l\textquotesingle article de Monsieur Papa Cheikh S. Sakho Jimbira, paru
le 16 Janvier 2022 révèlent que chaque année, on enregistre entre
100.000 et 260.000 jeunes sur le marché du travail. Le taux de chômage
global, estimé à 49 \% selon l\textquotesingle Agence pour le niveau
national, grimpe à 61 \% pour les moins de 30 ans en 2015. Ce chiffre a
connu une baisse ces dernières années mais reste quand même important.
Le chômage des jeunes suit une croissance exponentielle et ce depuis
plusieurs années. A côté des jeunes qui intègrent des emplois précaires
à la suite d\textquotesingle un échec scolaire, beaucoup de jeunes
diplômés sont également au chômage, faute de pouvoir trouver un emploi.
Même pour obtenir un stage, les refus sont fréquents. Ces freins sociaux
tiennent en grande partie à l\textquotesingle inadéquation entre
l\textquotesingle offre et la demande, au manque de compétences et
d\textquotesingle expérience mais aussi à la problématique de
l\textquotesingle absence de qualification professionnelle, qui pose la
question du fossé existant entre l\textquotesingle offre de formation et
les exigences du monde du travail, sans oublier la faiblesse du secteur
privé.

\ul{\textbf{PROBLEMATIQUE}}

La question de l\textquotesingle emploi des jeunes ainsi que celle de la
migration sont devenues des problèmes de plus en plus compliqués à gérer
par le gouvernement sénégalais. Plusieurs solutions ont été mise en
place pour limiter celles-ci mais force est de constater que ces
dernières ne font qu\textquotesingle augmenter. Le Plan Senegal
Emergent, l\textquotesingle un des piliers des politiques sénégalaises
pour l\textquotesingle avenir développé en 2014 pour une durée de 20 ans
vise d\textquotesingle installer l\textquotesingle économie sur une
trajectoire de croissance forte, inclusive, durable, créatrice
d\textquotesingle emplois. Les premiers résultats de la première phase
de ce plan ont été plus que convaincante puisqu\textquotesingle il a
permis entre 2014 et 2018 de créer près de 29 000 emplois
d\textquotesingle après le Groupe de la Banque Africaine de
Développement. A l\textquotesingle heure du bilan de la première phase,
il a été donc noté une création d\textquotesingle emplois croissante
mais encore insuffisante pour absorber la demande (\emph{Rapport Plan
Senegal Emergent, Plan d\textquotesingle Action Prioritaires
2019-2023}). Néanmoins, depuis les événements de mars 2021 qui ont
secoué les tissus social, politique et économique du pays,
l\textquotesingle emploi des jeunes semble constituer un mot ou une
expression à la mode. En effet, au conseil des ministres du 10 Mars
2021, l\textquotesingle accent a été mis sur le trytique
jeunesse-emploi-financement. Ainsi, Le chef de l\textquotesingle Etat,
en faisant référence à l\textquotesingle accélération de la relance de
l\textquotesingle économie nationale, de
l\textquotesingle intensification de l\textquotesingle exécution du
PSE/Jeunesse, du financement et de l\textquotesingle emploi des jeunes,
indique que le PAP2A/PSE et le budget de l\textquotesingle Etat vont
être profondément revus au regard des nouveaux impératifs, enjeux et
urgences signalés. Ces derniers tournent autour de la réorientation des
priorités autour de la jeunesse. Ainsi, ce sont au moins 350 milliards
de francs CFA qui vont être mobilisés dans la période 2021-2023 pour le
financement des jeunes et des femmes.

Face à cette situation alarmante et préoccupante de
l\textquotesingle emploi des jeunes au pays, nombreux sont ceux qui
préfèrent émigrer à la recherche d\textquotesingle emploi, espérant
changer leur condition. Certains obtiennent gain de cause et
d\textquotesingle autres quand bien même ne trouve pas un emploi
convenable à leur formation initiale arrivent quand même à
s\textquotesingle insérer dans le marché du travail et gagner leur vie.
L\textquotesingle on serait tenté alors de penser que le chômage au pays
d\textquotesingle origine est le principal déterminant de
l\textquotesingle émigration des jeunes sénégalais en quête
d\textquotesingle un emploi. Cependant aucune étude propre au Sénégal
n\textquotesingle a, à notre connaissance, statué sur les principaux
facteurs qui impulsent la migration motivée par la quête
d\textquotesingle emploi. Dès lors, il devient alors pertinent de
s\textquotesingle interroger sur la nature véritable des déterminants de
l\textquotesingle émigration des jeunes sénégalais pour de
l\textquotesingle emploi.

\ul{\textbf{INTERET DU SUJET}}

Cette étude sera intéressante en ce sens qu\textquotesingle elle
permettra non seulement d\textquotesingle appréhender la question de
l\textquotesingle émigration su Sénégal mais plus encore celle pour de
l\textquotesingle emploi. Ainsi elle apportera une plus-value dans
l\textquotesingle orientation des politiques pour
l\textquotesingle emploi des jeunes en vue de limiter ce fléau. En
effet, pour arriver à pallier le problème d\textquotesingle émigration
des jeunes en quête d\textquotesingle emploi, il serait plus judicieux
de créer un environnement favorable pour eux dans le pays
d\textquotesingle origine et cela ne peut être fait sans connaitre les
principaux facteurs qui motivent cette émigration. C\textquotesingle est
ainsi sur cette base que pourront voir jour des politiques dans ce sens.

\ul{\textbf{OBJECTIFS ET HYPOTHESES}}

L\textquotesingle objectif de l\textquotesingle étude sera ainsi de
faire ressortir les principaux déterminants de
l\textquotesingle émigration des jeunes sénégalais pour de
l\textquotesingle emploi. Elle repose principalement sur
l\textquotesingle hypothèse selon laquelle les jeunes migrent à la
recherche de meilleures conditions de vie à cause de leur situation
professionnelle qui est principalement chômeur.

\ul{\textbf{PLAN DE TRAVAIL}}

Ainsi, notre étude tentera dans un premier temps de présenter le profil
des émigrants pour de l\textquotesingle emploi. Elle se chargera par la
suite de déterminer les facteurs explicatifs de la migration des jeunes
sénégalais à la recherche d\textquotesingle emploi. Pour finir elle
tentera de proposer eu égard aux principaux résultats de proposer des
approches de politiques en vue de limiter cette émigration des jeunes
pour de l\textquotesingle emploi.

\newpage{}

\hypertarget{presentation-de-la-structure-daccueil}{%
\section{PRESENTATION DE LA STRUCTURE
D\textquotesingle ACCUEIL}\label{presentation-de-la-structure-daccueil}}

\hypertarget{pruxe9sentation-de-lansd}{%
\subsection{Présentation de l'ANSD}\label{pruxe9sentation-de-lansd}}

\hypertarget{historique-de-lansd}{%
\subsubsection{Historique de l'ANSD}\label{historique-de-lansd}}

Branche des mathématiques qui a pour objet la collecte, le traitement et
l\textquotesingle analyse des données relatives à un ensemble
d\textquotesingle objets, d\textquotesingle individus ou
d\textquotesingle éléments, la Statistique constitue un outil précieux
d\textquotesingle aide à la décision. En effet, que ce soit dans la
prise de décision, l\textquotesingle élaboration des politiques
publiques ou encore la suiviévaluation des projets pour
n\textquotesingle en citer que ceux-là, l\textquotesingle on fait
recours à la statistique qui se dote d\textquotesingle une importance
capitale au milieu de toutes ces opérations. Au Sénégal,
l\textquotesingle Agence Nationale de la Statistique et de la
Démographie (ANSD) est l\textquotesingle organisme officiel en charge de
la statistique. Créée par la loi N° 2004-21 du 21 juillet 2004, elle
fait office d\textquotesingle une structure administrative dotée
d\textquotesingle une personnalité juridique et d\textquotesingle une
autonomie de gestion. L\textquotesingle ANSD succède à la Direction de
la Prévision et de la Statistique (DPS). La loi de 2004 est venue
scinder la DPS en deux structures à savoir la Direction de la Prévision
et des Etudes Economiques (DPEE) et l\textquotesingle Agence Nationale
de la Statistique et de la Démographie (ANSD).
L\textquotesingle organisation de l\textquotesingle Agence et son mode
de fonctionnement sont régis par le décret N° 2005-436 du 23 mai 2005.
Elle a pour rôle d\textquotesingle assurer la coordination technique des
activités du Système Statistique National (SSN) et de réaliser elle-même
les activités de production et de diffusion des données statistiques
pour les besoins du Gouvernement, des administrations publiques, du
secteur privé, des partenaires au développement et du public.

\hypertarget{missions-de-lansd}{%
\subsubsection{Missions de l'ANSD}\label{missions-de-lansd}}

L'Agence Nationale de la Statistique et de la Démographie (ANSD) est une
structure administrative dotée de la personnalité juridique et d'une
autonomie de gestion et placée sous l'autorité du ministre chargé de la
Statistique. En particulier l'Agence est chargée :

\begin{itemize}
\item
  de veiller à l\textquotesingle élaboration et à la mise en œuvre des
  programmes pluriannuels et annuels d\textquotesingle activités
  statistiques ;
\item
  d\textquotesingle assurer la mise en application des méthodes,
  concepts, définitions, normes, classifications et nomenclatures
  approuvés par le Comité technique des programmes statistiques ;
\item
  de préparer les dossiers à soumettre aux réunions du Conseil national
  de la statistique et du Comité technique des programmes statistiques ;
\item
  d\textquotesingle assurer le secrétariat et
  l\textquotesingle organisation des réunions du Conseil national de la
  statistique et du Comité technique des programmes statistiques ainsi
  que de ses sous-comités sectoriels ;
\item
  de réaliser des enquêtes d\textquotesingle inventaire à couverture
  nationale notamment les recensements généraux de la population et les
  recensements d\textquotesingle entreprises ;
\item
  de produire les comptes de la nation ;
\item
  de promouvoir la formation du personnel spécialisé pour le
  fonctionnement du système national d\textquotesingle information
  statistique par l\textquotesingle organisation des cycles de formation
  appropriés notamment au sein d\textquotesingle une école à vocation
  régionale ou sous régionale intégrée à l\textquotesingle agence.
\end{itemize}

En outre l\textquotesingle Agence est chargée du suivi de la coopération
technique internationale en matière de statistique et représente à cet
effet le Sénégal dans les réunions sous régionaux, régionales et
internationales.

\hypertarget{organisation-administrative-de-lansd}{%
\subsubsection{Organisation administrative de
l'ANSD}\label{organisation-administrative-de-lansd}}

Placée sous la tutelle du Ministère de l\textquotesingle Économie, des
Finances et du Plan, l\textquotesingle ANSD est administrée par un
Conseil de Surveillance (CS) de représentants de la Primature, du
ministère de l\textquotesingle économie, des Finances et du plan, de la
Banque Centrale (BC), des organisations patronales, des centrales
syndicales des travailleurs et des centres de recherche universitaires.
Elle est également une composante du Système Statistique National (SSN)
qui comprend également le Conseil National de la Statistique (CNS) et
les services statistiques situés au sein des départements ministériels
et des organismes publics et parapublics. L\textquotesingle Agence est
dirigée par un Directeur général qui est assisté par un Directeur
général adjoint, nommé tous deux par décret sur proposition du conseil
d\textquotesingle orientation. Hormis la Direction générale à laquelle
sont rattachées la Cellule de Programmation,
d\textquotesingle Harmonisation, de Coordination Statistique et de
Coopération Internationale (CPCCI), la Cellule de Passation des Marchés
(CPM) et la structure d\textquotesingle audit interne,
l\textquotesingle ANSD est composée de six directions :

\begin{itemize}
\item
  Direction des Statistiques Économiques et de la Comptabilité Nationale
  (DSECN) chargée de l\textquotesingle élaboration des statistiques
  économiques et conjoncturelles, ainsi que de la production des comptes
  nationaux annuels et trimestriels ;
\item
  Direction des Statistiques Démographiques et Sociales (DSDS) chargée
  de la conception, de l\textquotesingle exécution et de
  l\textquotesingle analyse des enquêtes, des recensements
  démographiques, ainsi que des enquêtes socioéconomiques auprès des
  ménages ;
\item
  Direction de la Méthodologie, de la Coordination Statistiques et des
  Partenariats (DMCP) ;
\item
  Direction des Systèmes de l\textquotesingle Information et de la
  Diffusion (DSID) qui a pour mission d\textquotesingle assurer la mise
  à disposition d\textquotesingle un système
  d\textquotesingle information efficient pour
  l\textquotesingle ensemble des activités de l\textquotesingle ANSD et
  de gérer la diffusion des produits de l\textquotesingle Agence ;
\item
  Direction de l\textquotesingle Administration Générale et des
  Ressources Humaines (DAGRH) chargées de la gestion du personnel et des
  compétences de l\textquotesingle Agence, d\textquotesingle assurer sa
  sécurité sur toutes les questions juridiques et réglementaires, son
  approvisionnement et la gestion de la logistique et du matériel, elle
  gère les stocks de l\textquotesingle ANSD ;
\item
  Direction de l\textquotesingle École Nationale de Statistique et de
  l\textquotesingle Analyse Économique (ENSAE) qui a pour mission
  d\textquotesingle assurer la formation initiale et le perfectionnement
  des statisticiens pour les différentes composantes du SSN : les
  administrations publiques, les organismes publics et parapublics et le
  secteur privé. Elle forme des Ingénieurs Statisticiens Economistes,
  des Ingénieurs des Travaux Statistiques et des Analystes Statisticiens
  (AS).
\end{itemize}

\hypertarget{activituxe9s-ruxe9alisuxe9es}{%
\subsection{Activités réalisées}\label{activituxe9s-ruxe9alisuxe9es}}

Ce stage a également été l\textquotesingle occasion pour nous
d\textquotesingle appliquer les connaissances acquises à
l\textquotesingle école, d\textquotesingle approfondir certaines et de
se frotter aux réalités du monde professionnel. En effet, tout au long
des deux mois (Aout à Octobre) : Nous avons participé à plusieurs
activités entrant dans le cadre du RGPH-5 du Sénégal dont :

\begin{itemize}
\item
  La prise de connaissance de questionnaire du RGPH-pilote et ses
  objectifs : On s\textquotesingle est imprégner de cette phase du
  recensement qui vise essentiellement à tester les outils de collectes
  de données. Elle a concerné les districts de cinq régions tirés
  suivants des techniques de collecte de données ;
\item
  La participation aux différentes réunions de la coordination régionale
  de Dakar : Ces réunions ont pour but d\textquotesingle examiner les
  activités en cours notamment la cartographie et de formuler des
  recommandations face aux difficultés rencontrées par certains ;
\item
  La participation avec l\textquotesingle équipe de supervision de la
  cartographie dans la région de Dakar : Il s\textquotesingle agit pour
  l\textquotesingle équipe de s\textquotesingle imprégner des réalités
  du terrain des agents cartographes et d\textquotesingle échanger avec
  ces derniers aux problèmes rencontrés dans cette phase du recensement.
  Il faut noter qu\textquotesingle en 2013, la phase cartographique
  était déjà faite. Après 10 ans, il était important de mettre à jour
  ces informations vues les grandes mutations des hommes et
  l\textquotesingle évolution rapides des villes.
\end{itemize}

Pour notre part, on prend note et on fait une matrice de suivi qui est
transmise aux encadreurs pour validation. Cette matrice comprend
essentiellement trois colonnes (difficultés rencontrées, solutions et
recommandations).Nous avons aussi mis en pratique les enseignements
théoriques reçues à l\textquotesingle école et en parallèle faire du
lieu de stage une seconde école où on a pu acquérir de nouvelles choses.
Plus spécifiquement, le stage nous a permis :

\begin{itemize}
\item
  de consolider nos acquis et acquérir de nouvelles connaissances sur
  les logiciels STATA, Excel, R et SPSS ;
\item
  de connaitre une matrice de suivi ;
\item
  de comprendre les différentes phases du recensement ;
\item
  d\textquotesingle avoir quelques notions sur la migration dans sa
  globalité ;
\item
  de comprendre quelques notions en modélisation économétrique ;
\end{itemize}

\newpage{}

\hypertarget{cadre-conceptuel-et-revue-de-litterature}{%
\section{CADRE CONCEPTUEL ET REVUE DE
LITTERATURE}\label{cadre-conceptuel-et-revue-de-litterature}}

\hypertarget{duxe9finitions-des-concepts}{%
\subsection{Définitions des
concepts}\label{duxe9finitions-des-concepts}}

La migration est étudiée par les chercheurs de toutes les disciplines
des sciences sociales, ce qui a pour conséquence de multiplier les
approches dans sa définition et sa conceptualisation.
L\textquotesingle approche pluridisciplinaire de la migration en fait un
concept multidimensionnel qu\textquotesingle il est nécessaire de
définir afin de mieux organiser notre réflexion. En plus du concept de
migration, nous définirons des concepts qui lui sont dérivés et que nous
utilisons dans ce travail : la migration internationale, et le migrant.

\hypertarget{migration}{%
\subsubsection{Migration}\label{migration}}

Nous pouvons trouver une définition qui concilie les différentes
conceptions de la migration dans le Glossaire de la migration de 2007 de
l\textquotesingle Organisation Internationale pour les Migrations (OIM).
Selon ce glossaire la migration est « \emph{le déplacement
d\textquotesingle une personne ou d\textquotesingle un groupe de
personnes, soit entre pays, soit dans un pays entre deux lieux situés
sur son territoire. La notion de migration englobe tous les types de
mouvements de population impliquant un changement du lieu de résidence
habituelle, quelles que soient leur cause, leur composition, leur durée,
incluant ainsi notamment les mouvements des travailleurs, des réfugiés,
des personnes déplacées ou déracinées} ». La migration fait à la fois
appel à la dimension temporelle et spatiale. Ces deux dimensions sont
intrinsèquement liées. En effet, le mouvement dans
l\textquotesingle espace qu\textquotesingle elle opère ne peut être
qualifié de migration que s\textquotesingle il est fait sur une durée
bien déterminée. Toute la difficulté de conceptualiser la migration
réside non seulement dans la détermination de la distance parcourue et
de la durée minimum, mais aussi dans l\textquotesingle intention qui
motive ce déplacement. En effet, comment faire la différence entre un
long séjour touristique dans un pays différent et une migration. La
dimension temporelle fait référence à la notion de durée de la migration
mesurée par le temps écoulé depuis le changement de résidence. Dans de
nombreuses enquêtes démographiques, il est généralement retenu que cette
durée est d\textquotesingle au moins six mois pour qu\textquotesingle il
y ait migration. Cependant, d\textquotesingle autres auteurs réduisent
cette durée à 3 mois (Ba Cheikh, 1995 ; OIM, 2007 ; Guilmoto, 1997).
C\textquotesingle est ainsi qu\textquotesingle il est possible de
distinguer la migration de courte durée (plus de 3 mois à moins
d\textquotesingle un an) de celle de longue durée (au minimum un an). La
dimension spatiale de la migration fait intervenir la notion de
déplacement et de résidence. L'espace fait référence à un changement de
résidence, entendu par un changement de localité (village ou ville) qui
représente en fait la plus petite entité géographique du pays. Le
migrant se déplace d\textquotesingle un lieu d\textquotesingle origine à
un autre lieu et de ce fait, quitte sa résidence habituelle. La
résidence habituelle d\textquotesingle un migrant est le lieu dans
lequel il vit habituellement pendant une période minimum,
conventionnellement fixée à 6 mois. Par conséquent, dépendamment de la
durée de la migration, le migrant peut soit changer de résidence
habituelle ou tout simplement la quitter temporairement. De plus,
l\textquotesingle espace géographique d\textquotesingle où part le
mouvement migratoire et dans lequel il s\textquotesingle inscrit en
détermine le type. Il peut s\textquotesingle agir de migration
internationale lorsque le mouvement se fait au-delà de
l\textquotesingle espace national ou de migration interne
lorsqu\textquotesingle il se fait au sein de cet espace. Dans le cadre
de ce travail, nous allons plus nous intéresser à la migration
internationale.

\hypertarget{la-migration-internationale}{%
\subsubsection{La migration
internationale}\label{la-migration-internationale}}

La Migration internationale est définie comme « \emph{le mouvement de
personnes qui quittent leur pays d\textquotesingle origine ou de
résidence habituelle pour s\textquotesingle établir de manière
permanente ou temporaire dans un autre pays. Une frontière
internationale est par conséquent franchie} » (OIM, 2007). Cette
dernière prend le nom d\textquotesingle immigration et
d\textquotesingle émigration selon que le pays considéré constitue le
lieu de destination ou le lieu d\textquotesingle origine de ces
migrations.

Dans ce travail, la définition de la migration internationale que nous
prendrons sera la définition usuelle qui veut que la migration
internationale concerne le mouvement d\textquotesingle individus où il y
a franchissement des frontières internationales.

\hypertarget{migrant}{%
\subsubsection{Migrant}\label{migrant}}

Après avoir brossé le concept de migration, il nous faut définir le
concept de migrant. La définition du migrant donnée par
l\textquotesingle OIM (2007) est la suivante : « \emph{Migrant : Ce
terme s\textquotesingle applique habituellement lorsque la décision
d\textquotesingle émigrer est prise librement par
l\textquotesingle individu concerné, pour des raisons de convenance
personnelle et sans intervention d\textquotesingle un facteur
contraignant externe. Ce terme s\textquotesingle applique donc aux
personnes se déplaçant vers un autre pays ou une autre région aux fins
d\textquotesingle améliorer leurs conditions matérielles et sociales,
leurs perspectives d\textquotesingle avenir ou celles de leur famille}
».

Si ce concept décrit clairement le migrant comme un individu qui a
effectué une migration, il n\textquotesingle explique pas comment le
caractériser en vue d\textquotesingle analyses statistiques. Les
enquêtes démographiques auprès des ménages définissent les migrants de
manière concrète. Cependant ces concepts diffèrent parfois selon les
enquêtes en fonction des objectifs recherchés. C\textquotesingle est le
cas notamment des enquêtes que nous utilisons dans notre travail. Aussi,
nous présenterons les concepts des migrants tels que définis dans le
manuel du 5e Recensement de la Population et de
l\textquotesingle Habitat (RGPH-5), que nous utiliserons dans le cadre
de notre travail.

\begin{itemize}
\tightlist
\item
  \textbf{Définition tel que défini dans le RGPH-5}
\end{itemize}

Dans le RGPH-5, le migrant international ou encore
l\textquotesingle émigrant est le migrant qui a quitté son pays
d\textquotesingle origine (le Sénégal) pour s\textquotesingle installer
à l\textquotesingle étranger au cours des cinq années précédant le
passage de l\textquotesingle enquêteur dans le ménage et qui y demeure
encore. Le lieu de résidence antérieur de l\textquotesingle émigré est
le ménage enquêté déterminant ainsi son lieu d\textquotesingle origine.
Or, ce lieu ainsi spécifié n\textquotesingle est que le dernier point de
départ du migrant et non pas forcément la région dans laquelle il est né
ou dans laquelle il s\textquotesingle était établi durablement ou tout
simplement son véritable lieu d\textquotesingle origine. De plus, les
émigrations ne concernent que les sorties du ménage de six mois et plus.
Le schéma suivant permet de mieux saisir la période concernée :

Cette mesure ne permet d\textquotesingle obtenir des informations que
sur la migration actuelle récente, ce qui sous-estime le nombre de
migrants internationaux mais aussi le nombre de migrations. En effet, la
période de référence de 5 ans exclut toutes les migrations qui ont lieu
dans un passé plus lointain. En outre, la migration est un phénomène
renouvelable. Mais les migrants qui ont eu à quitter leur pays dans la
période de référence de 5 ans et qui ont eu à faire plusieurs
allers-retours ne sont pas comptabiliser ; d\textquotesingle où une
sous-estimation des migrations.

\hypertarget{revue-de-littuxe9rature}{%
\subsection{Revue de littérature}\label{revue-de-littuxe9rature}}

\hypertarget{revue-thuxe9orique}{%
\subsubsection{Revue théorique}\label{revue-thuxe9orique}}



\end{document}
